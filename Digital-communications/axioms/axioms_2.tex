\iffalse
\documentclass[journal,12pt,twocolumn]{IEEEtran}
\usepackage{graphicx}
\graphicspath{{./figs/}}{}
\usepackage{amsmath,amssymb,amsfonts,amsthm}
\newcommand{\myvec}[1]{\ensuremath{\begin{pmatrix}#1\end{pmatrix}}}
\usepackage{listings}
\usepackage{watermark}
\usepackage{titlesec}
\let\vec\mathbf

\titlespacing{\subsection}{0pt}{\parskip}{-3pt}
\titlespacing{\subsubsection}{0pt}{\parskip}{-\parskip}
\titlespacing{\paragraph}{0pt}{\parskip}{\parskip}
\newcommand{\figuremacro}[5]{
    
}
\lstset{
frame=single, 
breaklines=true,
columns=fullflexible
}
\thiswatermark{\centering \put(0,-105.0}{\includegraphics[scale=0.5]{iith.png}} }

\sloppy
\title{\mytitle}
\title{
Probability Assignment-II
}
\author{Nikhil Nair}
\begin{document}
\maketitle
%\tableofcontents
\bigskip



\section{\textbf{Problem }}
\fi
\solution
\begin{enumerate}
	\item 
		In this case
\begin{align}
\pr{A + B} &=\pr{A} + \pr{B}&
\\
\implies \frac{3}{5}&=\frac{1}{2}+p&
\\
\therefore  p &= \frac{1}{10}&
\end{align}
\item 
Given A and B are independent events,
then,

\begin{align}
\pr{A + B} &=\pr{A} + \pr{B} - \pr{A B}& 
\\
\implies \pr{A + B} &=\pr{A} + \pr{B} - \pr{A}\pr{B}&
\\
\implies \frac{3}{5}&=\frac{1}{2}+ p - \frac{p}{2}&
\\
\therefore p &= \frac{1}{5}&
\end{align}
\end{enumerate}

